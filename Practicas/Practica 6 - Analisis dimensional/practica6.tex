%\documentclass[10pt]{letter}
%\usepackage[utf8]{inputenc}

%%%%%%%%%%%%%%%%%%%%%%%%%%%%%%%%%%%%%%%%%%%%%%%%%
% compile with LuaLatex
%%%%%%%%%%%%%%%%%%%%%%%%%%%%%%%%%%%%%%%%%%%%%%%%%%%%%%%
\documentclass[11pt]{report}
\usepackage{epsfig}
\usepackage{amssymb,amsmath,amsfonts}
\usepackage[activeacute,american]{babel}
%\usepackage[utf8]{inputenc}
\usepackage{subfiles}
\usepackage{cite}
\usepackage{csquotes}
\usepackage{esvect}
\usepackage[acronym,nonumberlist]{glossaries}
\renewcommand{\acronymname}{Nomenclature}
\usepackage{multicol}
\usepackage{caption} 
\usepackage{float}
\usepackage[
    math-style=ISO,      % Upper Case Greek is in italics
    bold-style=ISO,      % Bold math is in italics
    partial=upright,     % nabla and partial upright
    nabla=upright,
  ]{unicode-math}
\topmargin 1.2cm 
\textwidth 16.1cm
\textheight 22.5cm
\oddsidemargin 0.7cm
\setcounter{tocdepth}{5}
\addtolength{\voffset}{-2.4cm}
\addtolength{\hoffset}{-0.5cm}
\usepackage{booktabs}
\usepackage{setspace}
%\doublespacing
\onehalfspacing
\usepackage{caption}
 \captionsetup[figure]{labelfont={bf},name={Figura},labelsep=period}


%%%%%%%%%%%%%%%%%%%%%%%%%%%%%%% 
% citas
% \footnotetext{Mott, Robert L. Mecanica de Fluidos 6/e. Pearson educación, 2006.}
% \footnotetext{Pritchard, Philip J. Fox and McDonald’s Introduction to Fluid Mechanics (8th ed.). John Wiley $\&$ Sons. (2011).}
% \footnotetext{Munson, Bruce R., et al. "Fundamentals of Fluid Mechanics, John Wiley $\&$ Sons." Inc., USA (2006).}
%%%%%%%%%%%%%%%%%%%%%%%%%%%%%%% 

%%%%%%%%%%%%%%%%%%%%%%%%%%%%%%%%%
\begin{document}
\centering{ \textbf{\Large{Mec\'anica de fluidos}}}

%\centering {\Large{2$^\circ$ semestre 2020: 541209-1}}
\vspace{1cm}

\flushleft{ \large \underline{\textbf{Pr\'actica 4: An\'alisis diferencial}}}

%%%%%%%%%%%%%%%%%%%%%%%%%
\vspace{1cm}

\underline {Problema 1 (P. 7.10 Fox\footnote{footnotes working fine}):}
\vspace{0.2cm}

Experimentos han demostrado que la caida de presi\'on para el flujo a trav\'es del orificio de diametro $d$ en una placa montada en una tuberia de diametro $D$ puede ser expresada como: $\Delta p = p_1 - p_2 = f(\rho,\mu,{V},d,D)$. Se le encomienda realizar algunos experimentos para determinar esta relaci\'on. Obtenga los grupos adimensionales resultantes.
\vspace{0.2cm}

\underline {Problema 2 (P. 7.22 Fox):}
\vspace{0.2cm}

La energ\'ia que se libera durante una explosi\'on, $E$, es una funci\'on del tiempo tras la detonaci\'on $t$, el radio de explosi\'on $R$ al tiempo $t$, la presi\'on ambiental $p$ y densidad $\rho$. Determine, mediante an\'alisis dimensional, la forma general de la expresi\'on para $E$ en funci\'on de las otras variables.
\vspace{0.2cm}

\underline {Problema 3 (P. 7.35 Fox):}
\vspace{0.2cm}

Pequeñas gotas de l\'iquido se forman cuando un jet de l\'iquido se separa en procesos de spray e inyecci\'on de combustibles. Se piensa que el diametro de las gotas resultantes, $d$, depende de la densidad del l\'iquido, viscosidad, tensi\'on superficial, velocidad $V$ y diametro $D$ del jet. ¿Cuantos grupos adimensionales son requeridos para caracterizar el proceso?. Determine los grupos adimensionales.
\vspace{0.2cm}

\underline {Problema 4 (P. 7.55 Fox):}
\vspace{0.2cm}

Los diseñadores de un globo meteorol\'ogico, cuyo prop\'osito es recolectar muestras de poluci\'on atmosf\'erica, desean conocer la fuerza de arrastre a la que el globo ser\'a sometido. Se anticipa que la velocidad m\'axima del viento ser\'a de $5$\,m/s (es razonable suponer que el aire se encuentra a una temperatura de $20$\,$^\circ$C). Un modelo a escala de 1:20 se construye para realizar pruebas en agua a $20$\,$^\circ$C. 
\begin{itemize}
\item ¿Qu\'e velocidad de agua se requiere para modelar al propotipo?.
\item Si a esta velocidad la fuerza de arrastre medida en el modelo es de $2$\,kN. ¿C\'ual ser\'a la fuerza de arrastre correspondiente en el prototipo?.
\end{itemize}
\vspace{0.2cm}


\footnotetext{Pritchard, Philip J. Fox and McDonald’s Introduction to Fluid Mechanics (8th ed.). John Wiley $\&$ Sons. (2011).}%%%%%%%%%%%%%%%%

\newpage

\underline {Problema 5 (P. 7.65 Fox):}
\vspace{0.2cm}

Las caracter\'isticas fluido-din\'amicas de una pelota de golf son testeadas utilizando un modelo en un tunel de viento. Los parametros dependientes corresponden a la fuerza de arrastre, $F_D$, y la fuerza de sustentaci\'on, $F_L$, sobre la pelota. Los par\'ametros independientes deben incluir la velocidad \'angular, $\omega$ y profundidad de los hoyuelos, $d$. Determine par\'ametros adimensionales adecuados y exprese la dependencia funcional entre ellos. Un golfista profesional puede golpear la pelota a $V=75$\,m/s y  $\omega=8100$\,rpm. ¿C\'ual ser\'a el di\'ametro necesario del modelo para modelar estas condiciones en un tunel de viento, cuya velocidad m\'axima es de $25$\,m/s?. ¿Qu\'e tan r\'apido rotar\'a el modelo? (El diametro de una pelota de golf en EE.UU. es de $4.27$\,cm)
\vspace{0.2cm}


\underline {Problema 6 (P. 7.88 Fox):}
\vspace{0.2cm}

Una bomba centrifuga funcionando a una velocidad $\omega=800 rpm$ tiene los siguientes datos para flujo volum\'etrico $Q$ y diferencia de presi\'on $\Delta p$:
\vspace{0.5cm}

\begin{center}
\begin{tabular}{l*8c}
\toprule
$Q$\,(ft$^3$/min) & 0 & 50 & 75 & 100 & 120 & 140 & 150 & 165 \\
\midrule
$\Delta p$\,(psf) & 7.54 & 7.29 & 6.85 & 6.12 & 4.80 & 3.03 & 2.38 & 1.23 \\
\bottomrule
\end{tabular}
\end{center}
\vspace{0.5cm}

La diferencia de presi\'on es una funci\'on del flujo volum\'etrico, velocidad, diametro del propulsor $D$, y densidad del agua $\rho$. Graf\'ique la diferencia de presi\'on vs. flujo volum\'etrico utilizando la informaci\'on previa. Encuentre los grupos adimensionales para este problema y graf\'iquelos. Real\'ice un an\'alisis num\'erico de las curvas y basado en este an\'alisis genere y grafique datos para diferencia de presi\'on vs. flujo volum\'etrico para velocidades del propulsor de $600$\,rpm y $1200$\,rpm.

%%%%%%%%%
%%%%%%%%%%%%%%%%%%%%%%%%%
%%%%%%%%%%%%%%%%%%%%%%%%%
\end{document}
